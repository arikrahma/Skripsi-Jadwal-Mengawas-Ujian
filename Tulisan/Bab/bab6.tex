\chapter{Kesimpulan dan Saran}
\label{chap:Summary}

\section{Kesimpulan}
Setelah melakukan penelitian maka dapat disimpulkan beberapa hal, yakni:
\begin{enumerate}
	\item \textit{Framework} ODOO mendukung untuk pembuatan ETL \textit{tool} yang berperan dalam proses BI.
	\item ODOO bersifat modular sehingga fleksibilitasnya tinggi.
	\item Jika SSIS mebutuhkan 2 \textit{tool} untuk ETL yaitu visual studio dan management studio, ODOO hanya membutuhkan 1 tool saja.
	\item Pengguna tanpa pengetahuan SQL dapat menggunakan \textit{tool} ini.
	
\end{enumerate}

\section{Saran}
Berikut ini saran yang diharapkan dapat membantu pengembangan penelitian ini lebih lanujut, yakni:
\begin{enumerate}
	\item Menambahkan \textit{input data} XML-RPC
	\item Menambahkan OLAP dan Cube
	\item \textit{Merge} yang pada saat ini baru bisa di jalankan apabila ditempatkan paling pertama, diharapkan pada penelitian selanjutnya dapat ditempatkan dimana saja.
\end{enumerate}